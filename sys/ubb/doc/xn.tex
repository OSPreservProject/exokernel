\documentstyle[timessyn,times,epsf,psfig]{acmconf}

\pagestyle{myheadings}
\thispagestyle{myheadings}
\newcommand{\xxx}[0]{{\sc XN}}

\title{Overview of the exokernel disk subsystem}
\author{Hi}
\begin{document}
\maketitle

\begin{abstract}

	This paper is intended to describe the \xxx\ disk subsystem in
	enough detail to write an efficient file system on top of it.
	It focuses on two aspects: the specific system call interfaces
	(and their associated semantics) used to interact with
	\xxx\ and the important policy decisions \xxx\ uses to
	guarentee disk integrity.  The former provide a fair degree of
	flexibility in terms of what data gets moved between core and
	disk, as well how it is represented, stored, and protected.
	Care has been taken so that the inflexibility required by the
	latter is contained.

\end{abstract}

\section{Introduction}

The goal of the \xxx\ interface is to be that of the disk itself.
Any movement away from this idealized interface are driven by the
need for protection.  In particular there are two requirements
of protection that prevent us from giving unlimited disk access
to applications (and, importantly, from using this access to implement
any abstraction in any way):
\begin{enumerate}
	\item The kernel must be able to do access checks for each
	block.  Currently, we just represent where the in a disk
	block are so that it can check them.  A more flexibile
	approach (that we will pursue) is to allow {\em methods}
	to be associated with each type of meta data.  These methods
	can be run to test whether a given principle has access
	to the block.  (As usual, we give security short shift
	for expedience.)
	
	\item The kernel must know what disk blocks are owned by an
	application.  To allow sharing of files between untrusting
	entities, we allow these blocks to be grouped.

	This requires either that the kernel tracks this information.
	essentially duplicating whatever file system the application is
	using and leaving little control for library file systems or
	that untrusted file systems can express what disk blocks their
	meta data controls. 

	Operationally, then, what we want is that given a piece of meta
	data the kernel knows what integers in it correspond to what
	disk blocks so that it can both prevent bogus modification and
	know what disk blocks a piece of meta data grants access to.
	This knowlege can be derived in two ways.  First, by building
	meta data from components that the kernel understands (disk
	block extents, byte arrays, etc.) thereby allowing it to guard
	them how it wishes.  The problem with components are that there
	are never enough of them.  Therefore we use a second approach:
	applications interpret their own meta data.

	This control lets them represent the data however they wish,
	unconstrained by preconcieved kernel notions.  To guard against
	malice and stupidity we force these interpretations to be
	representated using {\em UDFs} (Untrusted Deterministic
	Functions) that are tested for correctness using {\em ad hoc
	induction}.  This paper largely ignores UDFs.  The important
	intuition to keep in mind is that if a deterministic turing
	machine is started in the same initial state it will always
	compute the same output.  Therefore, but running UDFs before
	and after each modification to their state we can ensure that
	(1) the transistion was valid and (2) when we rerun the UDF the
	output of the latter run will be the same.

\end{enumerate}

\section{Implementation Goals}

There are three goals of the xn:
\begin{enumerate}
        \item Perform access control on the disk:  The single property xd
        must guarentee is that all disk block names in the meta-data
        are valid (or nil).  To do this the used to denote disk blocks
        always hold valid values (or nil).

        Key: ordered writes of meta-data.  We distinguish xn from db by
        reachability --- if a block can be reached by an xn then we assume
        it is one.  If we tagged using a value that could not appear in
        a disk block (or tied it with massive redundancy) then the ordering
        could be eliminated.

        \item Provide a cache coherence mechanism for disk: we do so
        via a cache registry.

        \item Allow applications to share meta data in a well-formed way.
        We do this using a notion of templates: operations on these
        templates are  guarded by the kernel, ensuring that no
        disk pointer or extent is set to garbage.  We may extend
        these protections to include a mechanism for constructing
        and binding methods to different portions of an xn.

        To provide strong atomicity guarentees we allow xn calls
        to be batched (these calls are automatically generated,
        and are in a different file).

\end{enumerate}

\section{Design Methodology}

The design philosophy is succinct: remove limists of ULFS freedom.  Ensuring
that unanticipated uses of \xxx\ are possible and efficient is most
easily solved by placing the machinery required for these uses in
untrusted code.  There are three concrete heuristics we use:
\begin{enumerate}

	\item Allocation, deallocation, and I/O operations are
	application initiated.  In some sense these are GOXP (``good
	old fashioned exokernel principles''); however, their
	realization has enough novelty to warrent mention.  In
	particular, \xxx\ {\em never} initiates an I/O request:  for
	any operation it is up to the application to make sure that the
	required data is in core and which data it is.  So, for
	example, when adding a disk block pointer to a piece of meta
	data if the meta data is not in core, the kernel returns an
	error code.  It is up to the application to refetch an try
	again.  Naively, this methodology seems to be misguided sloth;
	in reality, these operations are those that determine file
	system performance, therefore, having the application in
	complete control allows it to do what is required (rather than
	concentrate on tricking \xxx\ into {\em not} doing something
	that it wants).    
	
	For example, it may want to search through all blocks of 
	a file, but only those that are in core (e.g., if the file
	is for backing store and it is performing garbage collection).
	If \xxx\ initiates disk reads for any miss, then this operation
	will be tricky.  More prosaically, directory searches follow
	a similar pattern: scan all incore blocks for a name and if it
	is not there, fetch it from disk.  A large amount of this
	effect can be had using the exokernel principle of exposing
	data structures (since applications can check).  

	However forcing it, forces us to give applications the contorl
	they need and keep the implementation small and fast.   Adding
	default actions later may be done for conveneince.

	\item For a given function decompose it so that applications
	can tell \xxx\ what to do and how to do it, and it can check
	that the operations are legal.  Determining what and how 
	are where most of the action for file systems are; this 
	structuring places that action squarely in the applications
	provence.  (In a sense this method can be viewed as a
	case of Lampson's ``hints'' in that an expensive operation
	is performed by an untrusted entity, and \xxx\ merely check
	s it.)

	Concrete examples: \xxx\ guarentees that no persistent pointer
	gives access to uninitialized data.  Rather than determining
	what, when and which data to write to disk it merely records
	dependencies.  Applications then control how data is written
	and each write is checked to ensure it does not violate a 
	dependency.

	\item Applications can specify what they expect.  For instance,
	if they wish to free the backing store associated with a given
	disk block they may specify which page they anticipate it being
	associated with.  Using this control they are then able to 
	detect a variety of race conditions.
\end{enumerate}

\section{Architecture}

Artitecturally, \xxx\ is made of four main pieces.
\begin{enumerate}
	\item Template description table.  File systems describe their
	meta data (or reuse an existing description) using a template
	and store this template in the TDT.  Each type is named by an
	integer (its position in this table).

	Seperating the description of meta data from the meta data
	itself we obtain three main benefits.  First, since the
	description of meta data is seperated from the meta data
	itself, meta data can to be identical to that of a traditional
	file system running on the raw disk (e.g., traditional Unix
	systems).  Second, it allows meta data to be fairly rich, since
	self descriptive data structures impose harsh limits on what
	can be described concisely.  Finally, it allows us to associate
	pieces of code (methods) that are used to manipulate each
	type.  For instance, methods can be used to update modification
	time on inode modifications.  Similarly, methods can prevent
	applications from directly modifying the name field in inodes,
	allowing them to preserve uniqueness and name cache coherence.
	We make limited use of methods currently, in the future we will
	investigate using a combination of files and methods to replace
	servers with passive objects.

	\item  The root catalogue.  This tracks all roots file systems
	on \xxx .  To make a file system presistent, register its root
	in the catalogue and write that value of the catalogue out.  On
	reboot, the kernel can (logically) traverse file systems from
	these roots, garbage collecting and performing consistency
	checks.

	\item Buffer cache registry.    The registry tracks what disk
	blocks are in core, what types they contain, and the
	dependencies between them.  (Discussed more below.)

	\item UDF machinery.  UDF's are written in a deterministic
	pseudo-assembly language, interpreted in the kernel to detmine
	what disk blocks a piece of meta data guards.  UDF's return a
	set consisting of tuples of extents and types.  Before any
	modification of a piece of meta data, \xxx\ looks up its type
	in the buffer cache and finds its assocaited UDF.  It then runs
	the UDF, records the result, performs the requested
	modification, and then reruns the UDF, checking that the final
	state was a valid transition.

\end{enumerate}

\section{The Rules}

Ganger and Pratt summarize three rules for strict file system integrity:
\begin{enumerate}
      	\item Never point to a structure before it has been initialized.

        \item Never reuse a resource before nullifying all previous
        pointers to it.

	\item Never reset the old pointer to a resource before a new
	pointer has been set (when moving).

\end{enumerate}

Our interpretation of these rules is weakened in that we allow pointers
to uninitialized blocks that themselves contain no pointers.  This cuts
down dramatically on the amount and type of bookkeeping \xxx\ must do.
Additionally, file systems that are marked  as ``temporary'' (i.e., not
persistent across reboots) need not adhere to any of these precepts
since they will be wiped at reboot.  (\xxx\ tracks enough state so that
reboot is the only vulnerability.)

We consider each rule in turn below.

\begin{itemize}
	\item Rule 1.  In the current implementation this only applies
	to xn's marked as \v{XN\_PERSISTENT}.  Disk blocks and xn's marked
	as \v{XN\_TEMPORARY} can be read after a crash even if they are
	uninitialized (the xn's will not, however, be treated as truth.)  This
	weaker condition was provided (1) for simplicity and (2) because it
	seems to allow more sophisticated recovery techniques than otherwise.
	Note you can ensure no one can read your data by first overwriting it
	with zeroes (of course, offering the choice of fast or secure is
	inviting trouble.)

	Initialization is not a minor concern.  For example, weakly
	consistent file systems may want to avoid going to disk for
	most file creation and deletion operations.  A simplistic
	approach to uninitialized data will make this infeasible.
	
	At a high level, we allow non-persistent nodes to be written
	to disk in any order.  We only prevent the single write that
	would grant a tainted subtree persistence.

       	\item Rule 2. We guarentee the strong notion of this.

	\item Rule 3.  We guarentee this.  Duplicate pointers are
	eliminated at reboot by duplicating the data they point to.
	(We should either (1) allow the user to provide code that
	determines which block to keep or (2) mark trees as
	``unreconstructed'' so that applications can fix them.  The
	latter is easiest.)

\end{itemize}

Operations:
\begin{enumerate} 
	\item Volatile blocks are freed immediately.  If there are no
	persistent pointers to a block then it can be freed
	immediately.  If there are persistent pointers to it then we
	enqueue the block on the parent's  ``will free'' list  and,
	when the parent is written, walk this list freeing each block.
	If we did not have this guarentee then a file system that
	retained a persistent pointer to a freed disk block could
	access it after reboot.

	\item An xn can be written to disk iff there is no
	uninitialized xn that can be reached from it.  Implications:
	if an xn is ``unrooted'' (i.e., not reachable from a root in
	the root catalogue) then it can always be written.  If it is
	reachable, then we prevent any writes to it that will point to
	unititialized xns.  A file system can exploit this feature to
	perform any order of operations to an unattached subtree and,
	when done, attach it.  (Alternatively, could mark xn's as
	untrusted and in need of reconstruction...)

\end{enumerate}

{\em Expand on implementatation}

\section{Future tests}

	An interface is proven with use, not theory.  The following
are challenging implementations to place on top of \xxx :
\begin{enumerate}
      	\item A log structured file system

	\item Logging in general.

        \item A file system that inlines files in meta data.

        \item An xFS-like file system that uses bitvectors to
        determine allocation.

        \item A log-based file system that does not need our atomicity
        guarentees.

        \item A server-less name cache.
\end{enumerate}

\section{Architecture Restrictions}

There are three restrictions of the current implementation:
\begin{enumerate}
        \item Disk blocks can only be pointed to by a single entry.
                (will be able to eliminate with mm's.)

        \item Xn's can only be pointed to by a single entry (this
        shouldn't be all that bad to modify.  will be an attribute.)
                (will be able to eliminate with mm's.)

        \item We probably should add states to the registry to allow
        locking of various flavors.
\end{enumerate}


\section{UDF Overview}


As stated in the introduction, file system meta data is interpreted 
using UDF's.  UDF's are written in a pseudo-assembly language, and
imported into the kernel.  They can perform most machine operations
(with the exception of floating-point) and can access the given xn
using a set of stlylized instructions.  xn accesses are checked
Each UDF has a read and write set associated
with it that specify the byte ranges in the xn that it is allowed
to access.  Accesses not contained in these sets are errors and force
an error code.  The use of stylized instructions give the UDF system
enough semantic information to which makes checking these accesses simple.


In a fully general implementation, UDF's would be allowed to have
non-empty intersection in their read sets.  In this world, when a xn
update modified a byte in an xn all associated UDF's would be run.  In
the current implementation, UDF's must have non-overlapping read and
write sets.

Since the UDF system itself is not completely trivial, it is important
to keep in mind that it has a single, simple goal:  ensure that the
output of UDF's (i.e., the disk blocks and their types owned by an xn)
is trustworthy.  There are three requirements to implementing this goal
usefully:  (1) UDF execution is deterministic, (2) UDF are rechecked
iff state they depend on is modified.  (this seems obvious, but adds
constraints to multi-disk block uses of UDFs) and (3) there is only a
constant time overhead added from the use of UDFs.  This latter
constraint leads us to the use of {\em sub-partitioning}, which
essentially divides the xn into disjoint read and write sets (in a
sense, an xn is partitioned into multiple xn's that happen to reside in
the same disk block(s)).

Each UDF takes as input a pointer to an xn, and an index.  Indices
correspond to partitions formed during sub-partitioning and range
from $0$ to $n-1$ (where $n$ is the number of partitions).  Their
output is the set of typed disk blocks owned by this partition of
the xn.

UDFs are used as follows:
\begin{enumerate}
	\item Before a modification (e.g., allocation or deallocation
	of a disk block in an xn) the kernel runs the associated UDF.
	Its output is recorded.

	\item The given modification is then performed.  Modifications
	are expressed as a vector of $(offset, addr, nbytes)$ tuples
	that instruction where to write data in the xn.  Each tuple
	essientially corresponds to \v{memcpy(xn + offset, addr,
	nbytes)}.

	\item The UDF is then rerun.  The new set is compared to the
	old to ensure that the expected transition was performed.  On
	error the state modifications are rolled back.
\end{enumerate}

Actual code is given in Figure~\ref{Figure:UDF}.

\begin{figure*}

%%%grind -l c++
/* Specify modification. */
typedef struct xn_m_vec {
        size_t  offset;
        void    *addr;
        size_t  nbytes;
} xn_m_vec;

/* Unit of update. */
xn_update {
        db_t db;                /* base - all objects sector are aligned. */
        size_t nelem;           /* number of elements of t. */
        xn_elem_t type;         /* type of extent. */
};

/* Make a copy of the values to be modified by m. */
static void mv_checkpt(xn_m_vec *o, void *xn, xn_m_vec *m, int n) {
     int i;

     for(i = 0; i < n; i++)
           memcpy(o[i].addr, (char *)xn + m[i].offset, m[i].nbytes);
}

/* Write the modifications specifed by m into xn. */
static void mv_write(void *xn, xn_m_vec *m, int n) {
     int i;

     for(i = 0; i < n; i++)
          memcpy((char *)xn + m[i].offset, m[i].addr, m[i].nbytes);
}

/* 
 * Modify meta data xn using the modification vector m to give xn
 * control over the typed extent (db, n, type) specified in u.
 */
int alloc(void *xn, udf_fun *udf, int index, xn_m_vec *m, int n, xn_update *u) {
     Set old_s, new_s;
     xn_m_vec *om;

     mv_checkpt(om, xn, m, n);              /* Store initial state. */

     old_s = udf_run(xn, udf, index);       /* Get initial set. */
     mv_write(xn, m, n);                    /* Do mod */
     new_s = udf_run(xn, udf, index);       /* Get new set */

     /* old U (db, n, t) = new */
     ifset_eq(set_union(old_s, set_single(u)), new_s)) {
          return XN_SUCCESS;
     } else {
          mv_write(xn, om, n); 		/* roll back */
          return XN_ERROR;
     }
}
%%%end
\caption{Code to allocate disk blocks using UDF's.}
\label{Figure:UDF}
\end{figure*}

\section{Interface}

This section presents the current \xxx\ interfaces and some sample
uses, both correct (showing how to get something done) and not (showing
what invariants \xxx\ enforces).

At a high level a file system intializes the disk as:
\begin{enumerate}
	\item Its types are described using templates and stored in the
	type catalogue.

	\item  Its root (there may be several, but singular is
	easier) is stored in the root catalogue.  

\end{enumerate}

It can then use the system:
\begin{enumerate}
	\item It loads the types and root into memory (usually both
	will be cached).

	\item It can then allocate and deallocate blocks.  Allocation
	is done using a freemap (bit is on if block is free, off if
	allocated), which is reconstructed after a reboot.

	\item Its data is cached in a buffer cache registry.  The
	registry contains both meta data and disk blocks.  The registry
	is disk block based (i.e., each disk block produces a single
	entry, no matter how many types are contained in it).  Registry
	entries are used to track the state of each block (dirty, out
	of core, tainted, locked).  Importantly, an entry can be placed
	in the registry without requiring that the object it describes
	be in core (or have memory allocated for it).

	\item Each registry entry has an on disk location.  The
	contract with the kernel is that dirty blocks will be written
	to disk even if the application is not responsive.  This allows
	a simpler programming model than ensuring that an application
	is never swapped out when a revocation request arrives and also
	eliminates the need for a flurry of writes on process exit.
\end{enumerate}

The following subsections looks at (1) describing meta data using the
\xxx\ type system, (2) bootstrapping a file system, (3) building a file
system from the pieces provided.  Caveats, restrictions and wishes are 
strewn throughout.

\subsection{Describing xnodes}
Each type in a file system must be described to \xxx\ using the
\v{ubb\_type} structure.  For this paper, I will assume your meta data
has been already been expressed (since I'll do it).

The one function involved with the kernel is \v{sys\_xn\_import} which
checks the type for validity (primarily the code used for UDF's and
methods and any given offsets into an xn), and then writes the type
into its type catalogue.  Each type includes an application-specified
non-unique, non-guarded, character string that can be used to name
types.  This string could, for instance, be a secure hash of the fields
contained in the type.

To do:
\begin{enumerate}
	\item Making types be described in files that anyone can read and
	super-users can create.

	\item Ensure that multiple copies of types are not not around.
	
	\item Make type importantion more efficient (e.g., indirect
	blocks are very expensive to describe, since we create	
	a UDF for each of the 1024 entries).
\end{enumerate}

Applications must ensure that the disk blocks containing a type are
kept in core.  The kernel will return an error if the blocks are
paged out.  (The ``file system'' that implements the type catalogue
is rudimentary enough that applications should have no trouble 
traversing it.)  

The expectation with types is that there are only a few per file system
implementation.  For instance, an FFS file system would have five types:
directories, inodes, indirect blocks, double indirect blocks and triple
indirect blocks.  Obviously this assumption can be breathtakingly poor
if file systems require a a type per object; though even here,
if the type description is small compared object size then it doesn't
matter too much (e.g., as in a log structured file system).

Future work will focus on eliminating duplicate information between
types.  The main (current) restriction on types is that they must
evenly divide or be multiples of the disk block size.

The system call \v{sys\_xn\_import} imports a type into \xxx :
%%%grind -l c++
xn_err_t sys_xn_import(struct ubb_type *t);
%%%end
On success it returns the type name.  

(Unresolved issues: should you
have to specify the index into the type catalogue?  How to deal with
growth etc.?  Soln: unify into a ``catalogue'' file system that uses
methods, etc. to guarentee invariants.)

\subsection{Bootstrapping a file system}

File system trees are rooted in the root catalogue, which corresponds
roughly to the ``/'' directory of Unix.  Thus, boostrapping a file
system reduces to finding its root's index into the
catalogue.~\footnote{Should we provide space for file system specific
data here or just rely on implementations to store any that is needed
in the file pointed to by the root?.}

Since the catalogue is limited only by the size of the disk systems
must be prepared to page it in.  To make this simple, its structure,
layout and type names are codified in a well-known header file,
allowing file systems to perform the usual ``shouting down the hall''
needed to bootstrap any naming system.

Additionally, before a piece of meta data can be used the type template
for that meta data must be in core (otherwise \xxx\ cannot run its UDF's,
etc.).  Thus, the file system must be able to load these from disk into
the buffer cache.  Here too the structures, layout, and location of
the type catalogue are stored in a well known header file.

The disk blocks holding pieces of the root catalogue, the type
catalogue, and type templates are cached in the buffer cache.  We
expect that use will make bringing these pieces from disk infrequent,
usually following a system boot.

The following two calls can be used for bootstrapping:
%%%grind -l c++
/*
 * Allocate space for the root of an application-level file system.
 * Installs the type and pointer to the root in a ``well-known''
 * location on disk.  After allocation, applications can then alloc,
 * read, write, etc as normal.  On reboot, the kernel can (logically)
 * traverse file systems from these roots, garbage collecting and performing
 * consistency checks.  Fails if:
 *      - root is an invalid disk block or is already allocated.
 *      - type is an invalid type.
 */
xn_err_t sys_install_mount(da_t slot, db_t db, size_t nelem, xn_elem_t t);

/*
 * Return the disk block of the root catalogue.  Clients use this for
 * bootstrapping.
 */
db_t sys_root(void);
%%%end

{\em The biggest ``to do'' is that I need to actually store this information
on disk.  Currently, I keep it in main memory.  Uch.}

\subsection{Allocation and deallocation of disk blocks}


Allocation and deallocation have identical type signatures:  they
require the disk address of the parent (disk addresses are the byte
offset of the object), the update that is expected, and the
modification vector to use:  
%%%grind -l c++
/* Allocate the tuple specified in u. */
xn_err_t sys_xn_alloc(da_t da, xn_update *u, xn_modify *m);
/* Deallocate the tuple specified in u. */
xn_err_t sys_xn_free(da_t da, xn_update *u, xn_modify *m);
%%%end

The update and modify structures are as follows:
%%%grind -l c++
/* Specifies how to modify xn. */
struct xn_modify {
        cap_t   cap;
        size_t own_id;          /* id of the udf to run. */
        size_t n;               /* number of elements in mv */
        struct xn_m_vec *mv;    /* base, len, ptr pairs */
};

/* A unit of update. */
struct xn_update {
        db_t db;                /* base - all objects sector are aligned. */
        size_t nelem;           /* number of elements of t. */
        xn_elem_t type;            /* type */
};
%%%end

Error conditions for all of these call are:
\begin{enumerate}
 	\item The capability does not work.
 	\item The blocks are not free or allocated.
	\item The parent is not in core.
	\item The UDF does not exist.
	\item The modification vector contains bogus data.
	\item The UDF returned the wrong data or referenced values
	outside of its read or write set.
	\item The disk blocks deallocated do not match those given
	in the update structure.
\end{enumerate}

In addition we provide both move and swap operations, these take
structures identical to those used for allocation and deallocation,
but for both source and destination:
%%%grind -l c++
/* Swap v1:u1 with v2:u2. */
xn_err_t sys_xn_swap(da_t v1, xn_update *u1, xn_modify *m1, 
			da_t v2, xn_update *u2, xn_modify *m2); 

/* Write the value in v2:u2 into v1:u1 */
xn_err_t sys_xn_move(da_t v1, xn_update *u1, xn_modify *m1, 
			da_t v2, xn_update *u2, xn_modify *m2); 
%%%end

The only difference between swap and move is that move expects
that source will be nil after the modification.

See the ``rules'' section for a description of the restraints on
writing files to and from disk.


Given the following piece of meta data:
%%%grind -l c++
/* User-level sort-of-unix inode. */
struct inode {
	da_t		da;	/* disk address of the inode. */
        /* permissions. */
        unsigned short uid;
        unsigned short gid;

        /* important times: when last read, written and created. */
        unsigned mod;
        unsigned read;
        unsigned creat;

        /* pointers. */
#       define DIRECT_BLOCKS 15
        db_t db[DIRECT_BLOCKS];
        struct s_indir *s_indir;
        struct d_indir *d_indir;
        struct t_indir *t_indir;
};
%%%end
A sample use of the above routines is in Figure~\ref{Figure:Alloc}.

\begin{figure*}

%%%grind -l c++
/* Example of how to allocate and deallocate disk blocks.  */

/* Sugar for initializing the various data structures. */
void up_new(xn_update *up, db_t db, size_t n, xn_elem_t t)
    { up->db = db;  up->nelem = n; up->type = t; }

void mv_new(xn_m_vec *mv, size_t offset, void *addr, xn_elem_t type)
    { mv->offset = offset;  mv->addr = addr; mv->type = type; } 

void mod_new(xn_modify *m, cap_t cap, int udf_id, xn_m_vec *mv, size_t n) 
	{ m->cap = cap; m->own_id = udf_id; m->n = n; m->mv = mv; }

/* Used by dealloc and alloc. */
xn_err_t xn_realloc(xn_err_t (*op)(), struct inode *in, int n, db_t db, int t) {
	xn_update up;
	xn_modify m;
	xn_m_vec mv;

	/* Specify what to free/allocate. */
	up_new(&up, db, 1, t);

	/* Where to modify. */
	mv_new(&mv,offsetof(struct inode, db[n]), &db, sizeof db);

 	/* What cap and UDF to use to perform modification. */
	mod_new(&m, i->uid, n, &mv, 1);

	return op(in->da, &up, &m);
}

/* 
 * Routines to allocate/free direct blocks.  the UDF interpreting this
 * slot is named n.  ``in'' is a read-only copy of this value in the
 * kernel's buffer cache.
 */
int ffs_dalloc(struct inode *in, int n, db_t db) 
	{ return x_realloc(sys_xn_allocate, in, n, db, XN_DB); }

int ffs_dfree(struct inode *in, int n) 
	{ return x_realloc(sys_xn_free, in, n, in->db[n], XN_DB); }
%%%end

\caption{Example use of \xxx\ allocation and deallocation routines.}
\label{Figure:Alloc}
\end{figure*}

Using the \xxx\ interface is wordy since not only must you specify what
to allocate, you must specify how to do so, and what UDF is reponsible
for that portion of memory.  It may be worthwhile to use methods for
these modifications so that specifying an update is simpler.  Note,
however, that at some level this level of low-level modification must
be expressed since \xxx\ by design has not the first clue as to how
meta data is represented, modified, or interpreted.

Todo:
\begin{enumerate}
	\item Add reference counts and hard links.

	\item Test swap and move.  Currently not much has
	been done.

	\item Install methods for deleting two pointers to the
	same disk block.  (If move must be distinguished from 
	swap at reboot then we need to enforce an ordering.)

\end{enumerate}

\subsection{In core management}

Data manipulation includes: paging disk blocks in and out of the
registry, reading and writing data that is in the registry,
and creating and deleting registry entries.

Reading in a disk block requires reading in all blocks from
the root down to it.  These registry entries (though not the
disk blocks they correspond too) must remain in core for the
duration of the block's usage.  Assuming its parent is incore
reading a disk block consists of the following operations:
\begin{enumerate}
	\item Allocate backing store using \v{sys\_xn\_bind}.
	\item Allocate a registry entry using \v{sys\_inster\_attr}.
	\item Read in the block using \v{sys\_xn\_readin}.
\end{enumerate}
To write the block and then flush its storage:
\begin{enumerate}
	\item Write the bytes in the block using \v{sys\_xn\_writeb}.
	\item Write the block to storage using \v{sys\_xn\_writeback}.
	\item Deallocate the backing store and registry entry 
	using \v{sys\_xn\_unbind}.
\end{enumerate}


The interface for reading and writing unmapped bytes of data:
%%%grind -l c++
/* Copy [src, src + nbytes) to [da, da + nbytes).  */
xn_err_t sys_xn_writeb(da_t da, void * src, size_t nbytes, cap_t cap);

/* Copy [da, da + nbytes) to [dst, dst + nbytes).  */
xn_err_t sys_xn_readb(void * dst, da_t da, size_t nbytes, cap_t cap);
%%%end
(\v{da} is the disk address of the parent xn.)

These two routines fail if:
\begin{enumerate}
 	\item The capability is insufficient for any block $[db, db + nbytes)$.
 	\item Any block is out of core.
 	\item A page in the range is invalid.
 	\item $[dst, dst + nbytes)$ is not writable.
\end{enumerate}


The four routines for paging disk blocks between memory and disk:
%%%grind -l c++
/* Write [db, db + nblocks) back to disk.  */
xn_err_t sys_xn_writeback(db_t db, size_t nblocks, xn_cnt_t *cnt);
xn_err_t sys_xn_writebackv(struct xn_iov *iov);

/* Read the range [db, db+nblocks) from disk.  Fails if: */
xn_err_t sys_xn_readin(db_t db, size_t nblocks, xn_cnt_t *cnt);
xn_err_t sys_xn_readinv(struct xn_iov *iov);
%%%end

Note that these routines do not require permission checks; neither do
they require that any particular node is in core before the
read can happen.  This allows unauthenticated prefetching, which
can be handy.

These routines fail if:
\begin{enumerate}
 	\item Any db is invalid.
	\item (Read) There is no physical page to read the values into.
\end{enumerate}

Binding and unbinding pages to disk blocks is done using the following two
routines:
%%%grind -l c++
typedef enum { 
	XN_BIND_CONTENTS, 
	XN_FORCE_READIN, 
	XN_ZERO_FILL, 
	XN_ALLOC_ZERO, 
	XN_PREFETCH
} xn_bind_t;

/* Associate ppn with db. */
xn_err_t sys_xn_bind(db_t db, ppn_t ppn, cap_t cap, xn_bind_t flag, int alloc);

/*
 * Free page associated with db.  If ppn > 0 then it checks to
 * ensure that the given page is associated with the registry
 * entry.
 */
xn_err_t sys_xn_unbind(db_t db, ppn_t h_ppn, cap_t cap);
%%%end
If the \v{alloc} boolean is set, then we attempt to allocate page
\v{ppn} ($ppn < 0$ is interpreted as allocation of a random page).

There are five possible ways to bind a page to a disk block (using \v{flag}):
\begin{enumerate}
	\item Using the current page contents
	(\v{XN\_BIND\_CONTENTS}).  This requires write permissions.

	\item Using the given page, but require a ``readin'' from disk
	to initialize it (\v{XN\_FORCE\_READIN}).  Only requires read
	permissions.

	\item Bind and zero fill page (\v{XN\_ZERO\_FILL}).  Requires
	write permissions if the page has an on disk copy, otherwise
	only needs read permissions.

	\item If page is uninitialized, bind zeros, else bind using
	readin (\v{XN\_ALLOC\_ZERO}).  This combines the read case of
	(2) and (3).

	\item (Unsupported) \v{XN\_PREFETCH} : allocate backing
	store for an untyped disk block.  While the block can
	be paged in from disk, before it can be used it must
	be associated with a parent.
\end{enumerate}

Todo:
\begin{enumerate}
	\item Allow for mapping into address space (needed).

        \item Split nodes to allow for non-page aligned access.

	\item Provide batching to make these operations more efficient.

	\item Allow for some number of blocks to be pinned in memory
	or, alternatively, that code is called before a block is
	flushed to disk.


	\item Need to percolate disk block capabilities from the parent down.
\end{enumerate}

Creation and deletion of registry attributes us the following
routines:
%%%grind -l c++
/* Create an entry for the page described in op. */
xn_err_t sys_xn_insert_attr(da_t da, struct xn_op *op);
/* Delete a registry entry. */
xn_err_t sys_xn_delete_attr(db_t db, size_t nblocks, cap_t cap);
%%%end

These routines fail if:
\begin{enumerate}
 	\item The entry is guarding a violated constraint.
 	\item The entry is in use.
\end{enumerate}

\section{Shared state}

We intend to use methods to eliminate servers.  Servers are a
heavy-weight abstraction that prevent (invitably) application
freedom and are clumsy to use (e.g., most applications use
libraries rather than servers).

What do you need for well formed updates to shared state:

\begin{enumerate}
	\item Well-formed updates - don't want state to become
	meaningless.

	\item Persistence - want an object to live across process
	invocations (e.g. a name cache)

	\item Paging - don't want to have to pin the entire object in
	memory, since this makes shared data structures very expensive
	(again, think a large name cache)

	\item Logical segmentation - want to protect byte-ranges that
	may not, necessarily, be page granularities.  (e.g., a process
	table entry).

\end{enumerate}
Methods provide a way to do transparent operations in the presence
of software that doesn't understand.

xn is single threaded, which makes it way simple.

(Note the use of consensus instead of privilege: we agree to 
use a piece of authenticated code.)

\section{Todo}
\begin{enumerate}
	\item Go through the various comments and notes of the previous
	implementations and include them.

	\item Implement a simple, correct, file system that illustrates
	uses and concepts.  Main challenge will have handling shared state
	cleanly and concisely.
\end{enumerate}

For registry:
\begin{enumerate}
      	\item Add extent support
        \item Allow for dags - simple use of refcnts should do it.
        \item Need to add LRU machinery.
        \item See if we actually need states.  reachable seems redundant.
        \item Add \v{NO\_INTERNAL\_PTR} to each xr.

	\item Self descriptive UDFs.  The partitioning should
	not be done using partitioned sets.
	\item Test handling of non disk block sized types.

\end{enumerate}

% \section{A simple file system}
% \bibliography{biblio}
% \bibliographystyle{plain}
\end{document}
