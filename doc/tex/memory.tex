\section{Memory}

This section describes how the kernel multiplexes physical memory. At
a later time page revocation and how ExOS implements paging will be
added.

\subsection {Xok and physical pages}

Each physical page is described by a {\tt Ppage}
structure.\footnote{See sys/kern/pmap.c and sys/xok/pmap.h for
details} This structure tracks the following key information about the
page:

\begin{itemize}
\item whether the page has been allocated by the kernel, a user level
process, or if it's free.

\item whether the page is pinned for pending DMA

\item whether the page contains a disk buffer

\item what capabilities are associated with the page via an ACL
\end{itemize}

Pages are reference counted if they are allocated to a user-level
process.  The kernel can hold references to such pages, but the only
way a user-level process can hold a reference is by mapping the
page. When the last mapping to a page is removed the page is
considered free-able.

Process call {\tt sys\_insert } to modify the x86 page table
associated with the environment. To unmap a page, insert a pte without
the PG\_P (present bit) set. To map a page, insert a pte that points
to a physical page that is not allocated (to allocate it) or to a
physical page that is allocated and which the caller presents an
appropriate capability for.

\subsection {Shared data structures in ExOS}

In order for libos's to communicate with each other and share state
(such as fd's across forks and the mount table), shared memory is
used. Currently, sysv shared-mem interfaces are used to provide
this.\footnote{see lib/libexos/os/shared.c} Due to limitations in our
sysv implementation the first process booting must allocate the
segments that are going to be used and then other processes may attach
them as needed.

Shared memory is being used less and less since it does not give the
fault isolation that is typically desired. However, it is interesting
to note that while developing ExOS we never ran into an occasion of a
buggy libos crashing the system by scribbling bogus data into a shared
memory region. The problem we did run into, though, was version
control.  Every time we changed the structure of the shared memory
region every application had to be updated to understand the new
layout.

\subsection {Memory layout}

Processes load at 800000 and their entry point is 800020 (to skip the
initial a.out header). Their code, data, and heap all follow. Then
staticly reserved blocks of memory begin.\footnote{see
lib/include/exos/vm-layout.h for details} After this, optionally, is a
read-only mapping of the shared library followed by the private
write-able data and heap for the shared library.

Growin down from the high-end of the address space is the kernel which
maps all of physical memory high in the address space, then the
kernel, kernel data, etc.\footnote{see sys/xok/mmu.h} Then comes the
kernel readonly data structures and finally an unmapped page to
delineate between kernel addresses and user-addresses. At the upper
end of the user part of the address space is the user stack.

\subsection {Page faults}

When the kernel detects a page fault of any kind (writing to RO page,
access page not present, etc) it propagates the fault up to a user
level fault handler that.\footnote {see lib/libexeos/os/fault.c} ExOS
uses this facility to demand-page itself in and to dynamically grow
the stack down as needed as well as to handle copy-on-write faults.
