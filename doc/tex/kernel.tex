
\section {General xok Kernel Stuff}

This section describes a number of aspects of the xok kernel that apply
to many of its subsystems.

\subsection {Adding System Calls}

Adding system calls to the xok kernel is very easy.  To do so, one simply
writes the code for the system call and adds one line to
{\tt sys/conf/syscall.conf}.  Scripts invoked by the makefiles will
re-generate the appropriate header files and stubs in order to export
the new system call.

The exact format of the lines in {\tt sys/conf/syscall.conf} is as follows
(illustrated by example):

\centerline{\tt 0x78    loopback\_xmit   int, struct ae\_recv *}

where 0x78 is the system call number, {\tt loopback\_xmit} is the name of
the system call, "int" is the return type of the system call, and
"struct ae\_recv *" is the argument list for the call.  Some notes:
\begin{itemize}
\item the system call number must be unique (among those specified in
      {\tt sys/conf/syscall.conf}), which is why we keep it sorted by
      convention.
\item the actual system call name (both as defined in the kernel and as
      defined at user level) will be augmented with "sys\_".  So, the
      example above specifies {\tt sys\_loopback\_xmit}.
\item the system call number (as an "unsigned int") will be prepended to
      the argument list for the system call.  The auto-generated stubs will
      provide this extra argument (so application writers do not have to know
      about it), but kernel code writers must accept it in their system
      call implementations.
\item the argument list for the system call can be augmented by adding
      more comma-separated types.
\end{itemize}

\subsection {Device Drivers}

\subsection {Capabilities}

\subsection {Environments}


XOK's idea of a process is called an environment\footnote{see
sys/kern/env.c and sys/xok/env.h for details}. An environment is
fairly minimal: it is simply a key-ring of capabilities, a list of
up-call entry points where control should be diverted when interesting
events occur, a page table, and a collection of accounting info about
the process.  Further, each library operating system is allowed to
store whatever data it finds convenient into the environment's
``u-area''.

The primary calls for manipulating environments are {\tt
sys\_env\_alloc} and {\tt sys\_env\_free}.

An environment's page tables must be filled in explicitly and CPU time
must be allocated before it will be run. See the sections on memory
and scheduling for information on how to do this.


