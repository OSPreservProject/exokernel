
ExOS uses a VFS like interface between high-level file operations like
read, write, close, etc and the underlying filesystem object that
a fd represents.\footnote{see lib/libexos/fd/proc.c} Each underlying
object specifies an array of function pointers that implement the
specified operation allowing new filesystem objects to be transparently
used.

Several of these file object methods deserve particular attention
because they do not correspond to typical VFS-like operations:

\begin{itemize}

\item {\bf select\_pred}: This returns a wakeup predicate that will evaluate
to true when the object is ready for an operation. The top level fd code
collects wakeup predicates from all fd objects that a select is being
done on and concatenates them together to form one large predicate
that the process blocks on.

\item {\bf lookup}: This implements a partial namei. It maps a single
pathname component from one file object to another. The source and
destination file objects may be different types of file objects
if the pathname component corresponds to crossing a mount point.

\item {\bf bmap}: Translates a logical block number to a physical
block number. Both block numbers are only meaningful to the
underlying fd object. For example, an NFS fd object might implement
this function as the identity function since NFS has no concept of
physical block. This function is used when programs wish to bypass the
normal filesystem operations and want to read/write an object's blocks
directly.

\item {\bf mount}: Associates a mapping from one file object to another.
Subsequent lookup operations at this point will return the new file
object.

\item {\bf close0}: Close operation called on the underlying object
every time close is called on the fd. The close method is only
called when the last reference to an fd is closed.

\end{itemize}
