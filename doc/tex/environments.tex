
XOK's idea of a process is called an environment\footnote{see
sys/kern/env.c and sys/xok/env.h for details}. An environment is
fairly minimal: it is simply a key-ring of capabilities, a list of
up-call entry points where control should be diverted when interesting
events occur, a page table, and a collection of accounting info about
the process.  Further, each library operating system is allowed to
store whatever data it finds convenient into the environment's
``u-area''.

The primary calls for manipulating environments are {\tt
sys\_env\_alloc} and {\tt sys\_env\_free}.

An environment's page tables must be filled in explicitly and CPU time
must be allocated before it will be run. See the sections on memory
and scheduling for information on how to do this.
