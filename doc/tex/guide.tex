\documentclass[11pt]{article}

\usepackage{times}
\usepackage{fullpage}

\title{Exopc Getting Started Guide}

\author{
Parallel and Distributed Operating Systems Group\\
\\
MIT Laboratory for Computer Science \\
545 Technology Square \\
Cambridge MA 02174 \\
\\
exopc-request@amsterdam.lcs.mit.edu\\
http://www.pdos.lcs.mit.edu
}

\begin{document}
\maketitle

\section{Requirements}

Exopc is not yet self-hosting. This means that it must be compiled
under a different operating system than itself. Currently, exopc must
be built under OpenBSD or Linux, though FreeBSD or NetBSD probably
work too. We use either OpenBSD 2.2 with gcc 2.7.2 or linux 2.0.x with
libc6 and gcc 2.7.2. Exopc uses the same binary format as
OpenBSD. This means that if you're compiling under linux you have to
use a cross compiler that generates OpenBSD binaries. We have included
binaries for gcc 2.7.2 setup to cross compile in this distribution. To
use this cross-compiler, untar tools/linux-cross/cross-tools-libc6.tgz
under /usr/local/openbsd-cross.

Booting is poorly supported right now. Either you have to boot with
OpenBSD's boot blocks (not supplied) or with the supplied DOS
boot-loader. OpenBSD's boot blocks expect the kernel to be in an
OpenBSD filesystem. Other boot-loaders should work. It would be nice
to find one that didn't require us to have an OpenBSD filesystem and
which could load both linux and OpenBSD kernels.\footnote{A
boot-loader called ``grub'' may be just this, but we haven't tried it
out yet}

Exopc currently only supports the following hardware: NCR 810/815/875
SCSI controllers and SMC Elite16, EtherEZ\footnote{You must configure
this card using the card's ezsetup.exe program to set it as follows:
I/O base=280, IRQ=2, RAM Addr=d0000, wait states=no, net connection=1,
rom disabled=yes}, EtherPower 10/100 ethernet cards (not made anymore)
and dlink-500TX's (or other tulip based cards). You have to have an
ethernet card, but a local disk is optional. At least a Pentium
processor is required. Any color display adapter should work.

\section{Building}

Building the system is pretty straight-forward. Exopc uses gmake and you
should be able to go into a directory and type ``gmake'' either by
itself or with the targets ``install'' or ``clean''. ``clean'' does
the normal thing while ``install'' copies files from the build area to
the root filesystem of the exopc machine. 

Remember that exopc mounts a root filesystem via NFS when it starts
up. ``install'' copies files into this root filesystem on the
server. In order to tell the makefiles where to copy files you have to
set the {\tt EXODEST} environment variable. This points to the top of
a directory-tree that will later be your exopc root directory. For
example, you may have /home/pinckney/exopc as the top of you exopc
source directory and point {\tt EXODEST} to
/home/pinckney/exopc-root. Then, when you run ``gmake install'' from
/home/pinckney/exopc, /home/pinckney/exopc-root will be populated with
subdirectories bin/, etc/, usr/ etc and binaries will be copied into
the correct places.

In the special case that your EXODEST points to a directory on the
same disk as you are building on, you can use symlinks instead of
``install''. Two variables in GNUmakefile.global, {\tt LNPROGINSTALL}
and {\tt LNLIBINSTALL}, controll whether ``ln'' or ``install'' is
used. By default, ``ln'' is used so if you're building and installing
on different disks you need to change this.

The following lists some caveats of the current build process:

\begin{itemize}
\item Technically ``gmake install'' should build any out of date programs
and then install them, but this doesn't always seem to work. To be
on the safe-side, do a ``gmake; gmake install''. 

\item Until you understand all the cross dependencies between different
parts of the system, do full builds of the entire source tree.

\item Most binaries are linked against the shared library libexos.so.
Thus, if you change any of the libraries, rebuild the shared library
in bin/shlib and then rebuild {\em all} of the binaries. 

\item The first program is special-cased and resides in ebin/. The
section on booting contains more information about selecting which of
these programs to use as the first process. In any event, the
environment variable INITPROG can be set to choose among rconsoled,
kbdinit, and bootp. If this program changes, you need to rebuild the
kernel in sys/ to relink the first program into the kernel.
\end{itemize}

\section{Booting exopc}

Exopc currently boots much like a diskless workstation. Because we do
not have crash recovery for local storage, we must "start from
scratch" each time the system starts. This means that there is no
local root partition. Once the kernel loads, an NFS directory is
mounted as the root of the exopc system. Later, a local filesystem
may be created and mounted.

The kernel is typically loaded using OpenBSD's boot blocks which means
the kernel itself has to be stored in an OpenBSD filesystem. Further,
since the kernel itself is fairly minimal, it has no concept of
filesystems. Thus the kernel cannot load the first program. Instead,
the first program is encoded into the kernel. On booting, the kernel
loads this first program from the kernel's data segment into a new
process and starts the process running. This first process is then
responsible for mounting the root partition via NFS and loading any
subsequent programs.

This first process needs to know certain information before it may
mount the root such as its ip address, the NFS root path and server
address, the netmask, which ethernet interface to use, and possibly a
gateway address. This information can be provided in three different
ways:

\begin{itemize}
\item{\bf keyboard:} you type in the relevant information each time
you boot the system. This is typically only used while you are
getting the system setup.
\item{\bf hardcoded:} you can hardcode the information into the first
process. 
\item{\bf BOOTP:} you can setup the first process to use BOOTP to
get all the relevant information.
\end{itemize}

When building the kernel you must select which of these three options
you wish to use. See the section below on building for more information.

Once the first program has been started and local configuration
information gathered, /etc/rc.local is run. This is typically a shell
script that starts various system daemons as well as typically
building and mounting a local filesystem. The default rc.local should
not need to be changed.

\subsection{Using the keyboard to enter boot-time information}

If {\tt kbdinit} is selected as the first process, it will prompt
you with the following questions:

\begin{verbatim}

Please Enter your IP Address:
Please Enter your netmask: 
Please Enter interface of this IP address [%s]: 
Please Enter your Gateway [press return if you don't care]: 
Please Enter IP Address of your NFS Server: 
Please Enter name your NFS directory: 

\end{verbatim}

The NFS directory you mount should be the same directory that is
pointed to by your EXODEST env variable.\footnote{(note that OpenBSD
machines want the exact directory that you are mounting to appear on
the /etc/exports file, or you can put the first directory on that
partition followed by the "-alldirs" flag such that ALL directories in
that partition are exported.} The building section has more information
on EXODEST.

\subsection{Hardcoding boot-time information}

This is the original method of configuring boot-time information and
remains primarily for compatibility.  For historical reasons the program
that implements it is called rconsoled. All local configuration info.
is stored in the file include/exos/site-conf.h. Most of the entries
are self-explanatory. The one that is not is {\tt MY\_IP\_MAP}. Each
line of it is of the form:

\begin{verbatim}
   /* 1 */ {"panam", {18,26,4,38}, {0x0,0x0,0xc0,0x81,0xe0,0xe2}, 1},
\end{verbatim}

The first field is the machine's name. The second field is the machine's
ip address. The third field is the ethernet address of the primary
interface. The fourth field is the interface number. The last entry
in this table should consist of all zero entries to mark the end.

You must have at least two entries in {\tt MY\_IP\_MAP}: one for the
exokernel machine itself and one for the root NFS server. The former
is looked up by its ethernet address and the latter by its name.

\subsection {Using BOOTP for boot-time information}

This is the preferred way of configuring the first process. It obtains
all information via the BOOTP protocol. To use it, enable the bootp
daemon on some machine by uncommenting the line that starts with
"bootps" in the /etc/inetd.conf file and restarting inetd by sending it
SIGHUP signal (e.g., "kill -HUP").

All the information about the machine's ip address, netmask, nfs root
is obtained via the BOOTP protocol.  To use it add an entry to your
/etc/bootptab\footnote{make sure you use the ip address of your nfs
server since DNS is not setup at mount time}. Here's an example of
what ours looks like:

\begin{verbatim}
avoch::ha=0000c0d0a54e:\
	:sm=255.255.255.0:gw=18.26.4.1:
	:ef="NFS 18.26.4.40:/disk/av0/hb/tree":\
	:ip=18.26.4.33:\
	:to=auto:
\end{verbatim}

\subsection{Reloading kernels over the network}

Once the system is up and running, a new kernel may be overlayed on top
of the existing one using the exokernel's {\tt loadkern} program over NFS.
Specifically, running {\tt loadkern <xok image>} from the command line
(on an exokernel system) will reboot the system, loading the specified
kernel image.  This utility removes the need to boot into
OpenBSD every time you build a new kernel and want to run it.  Just
leave your old kernel on the OpenBSD partition, the boot blocks will
load this old one, and when the system is up you can reboot it with
the new kernel over NFS.  (The same is true of other booting methods,
like via DOS or floppy.)

The three initial programs discussed above all will check to see
if the latest installed kernel is newer than the one that was booted.
If it is, these initial programs will automatically reboot using the newest
kernel image.  This works by checking /boot/xok.osid in the root NFS partition.
If the contents of this file don't match the os\_id of the running kernel,
then /boot/xok will be reloaded on top of the existing kernel.  If the file
/boot/.noboot exists, then this automatic reloading will be disabled.

Of late, we have observed iffy behavior in the auto-detection/negotiation
between SMC EtherPower 10/100 cards and a SMC TigerSwitch 100 when reloading
a second kernel during the boot process.

\subsection {Booting from a floppy}

The xok kernel can be booted from a DOS partition rather than an OpenBSD
file system.  However, this is not self-contained booting.  You still need
the rest of the xok files accessible over NFS.  The benefit of this method
is that you don't need an OpenBSD file system to store a copy of your xok
kernel on.  You can almost think of this as a network boot loader specially
for xok.

You'll need a DOS floppy with nothing on it except command.com and
whatever system hidden files go with it.  In other words, format a
floppy with DOS and run ``sys a:'' to transfer the minimum required DOS
files for booting DOS.

Then place ``boot.com'' in the root directory of the floppy.  This file is
provided by OpenBSD for booting OpenBSD kernels from DOS.  A copy has
been included in the tools/floppy\_boot directory of the xok distribution.

Next, once you've successfully compiled/built xok, place the kernel image
(sys/obj/xok) onto the floppy as well.  The common mechanism for doing this
on linux or BSD systems is to use the mtools utilities (e.g., mcopy).  At
the moment, the kernel is small enough to fit onto the floppy.  Hopefully,
it won't get too big to fit on there any time soon.  Actually, my latest
kernel is too big, but a small DOS partition on the hard disk works fine.

Now, if you boot with that floppy you can type ``boot xok'' at the DOS
prompt and xok should load.\footnote {Remember that once loaded it
will check in EXODEST/boot/ to see if there's a newer version of the
kernel and automatically load that instead}.

If you wish, you can create an autoexec.bat file on your floppy to
automatically run ``boot xok'' every time.  Simply create a file called
autoexec.bat in the root of the floppy with ``boot xok'' as the first
line.

\subsection {Misc Unix environment setup}

You may need to edit some of the files in tree/etc/, notably rc.local,
hosts.conf and passwd. Files under tree/ are copied over to exopc root
filesystem when you do a ``gmake install''. Because we use the OpenBSD
libc our password setup follows OpenBSD's. This means you will have to
add/modify accounts by editing master.passwd and then running
tree/etc/rebuild\_passwd. All of the passwords in that file are
currently ``exopc''. If you want to add a new home directory edit
directories/GNUmakefile to create one.

You need to change resolv.conf to point to your name server and to set
the domain search order. You also need to change rc.local to tell
gettime to get the time from one of your own computers rather than
ours.

\end{document}
