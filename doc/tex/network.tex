\section{Networking}

Most of the networking code resides at user-level. The user-level code
does two key things: managing routes/interfaces and implementing
tcp/ip. The tcp/ip implementation is beyond the scope of this
document.  Useful examples are in bin/webserver/httpd, bin/videod, and
test/xio-demo. The base tcp/ip stack itself can be found in
lib/xio. Note that the prime goal of this library is to facilitate
building specialized apps. This stack is structued as a series of
primitives that each perform a usefull function rather than a
monolithic stack.

bin/icmpd/icmpd.c is a fairly simple stand-along example of how to
write network programs. It uses DPF, packet rings, routes, interfaces,
and the packet sending interfaces.

\subsection{Kernel Interface}

The kernel implements two sets of interfaces: one for sending packets
and the other for receiving packets. On the receiving side, packets
are first demultiplexed to the proper owner and then buffered in
user-supplied memory in packet rings. 

\subsubsection {DPF}

Each process specifies what packets it's interested in receiving by
downloading a packet filter. Ideally, the most specific filter
gets a packet if filters overlap. Currently, however, this doesn't
quite work. It's not quite clear what happens with overlapping filters
at the moment.

Filters are specified by building up a collection of <offset, value>
pairs. The offset corresponds to byte offsets into an incoming packet
and value corresponds to the value that must be found. Here's an example
that matches IP packets coming in over ethernet:

\begin{verbatim}

    dpf_begin (&ir);
    dpf_eq16 (&ir, eth_offset(proto), htons(EP_IP)); /* eth proto = ip */
    dpf_eq32 (&ir, eth_sz + ip_offset(destination), ip);  /* ip dst = us */

    demux_id = sys_self_dpf_insert (CAP_ROOT, &ir, ringid);

\end{verbatim}

{\tt demux\_id} is a handle to the filter that can be used in future calls
to manipulate the filter. {\tt ringid} specifies the packet ring that
matching packets should be copied into. See the following section for
more details about packet rings.

\subsubsection {Packet Rings}

Packet rings are a ring of pointers to user-supplied buffer
space. Incoming packets for an application are copied to the next
empty buffer. Each buffer in the ring is owned by either the kernel or
the application, depending on the value of a flag field in the buffer
struct. If this flag is zero the buffer is empty and can be filled by
the kernel. Non-zero values are the length of the packet that were
copied into the buffer. When a process is done with a packet it should
reset the flag to zero.

See bin/icmpd/icmpd.c for a complete example.

\subsubsection {Sending Packets}

{\tt ae\_eth\_send} is a simple interface to both the loopback, slow
ethernet, and fast ethernet drivers. It does not support
scatter/gather at the moment so for demanding apps this isn't what you
want. {\tt sys\_ed\_xmit}, {\tt sys\_de\_xmit}, {\tt
sys\_loopback\_xmit} are system calls for sending to each interface
and which support the full capabilities of each card. The xio tcp
stack is a good example of how to use {\tt
sys\_de\_xmit}.\footnote{See lib/xio/exos\_net\_wrap.c}

\subsection {Routes, Interfaces, DNS, ARP, etc...}

This section describes the misc. stuff involved in networking. XOK
supports multiple ethernet interfaces and exopc comes with the
typical ifconfig, route etc tools along with an arp demon and DNS
resolver.

\begin{itemize}

\item{\bf Interfaces}:
During boot up, the kernel initializes the network cards, and
the sysinfo structure.  The sysinfo structure contains the number of
slow and fast ethernet cards that the kernel recognized.  

The IP address of the machine will be determined by the first
process.  This process then proceeds to initialize a global interface
table that contains basic ``ifconfig'' information, like physical
interface number, ip address, ethernet address (cached from sysinfo),
netmask and flags.  Programs use this information to know if an IP
address is on the local LAN or remote; if it is on a remote location,
then they look in the route table for the route.

\item{\bf ARP}: The first process spawns an arp daemon.  There is a global 
arp\_table mapped into each process.  Only the arp daemon can write to
this table.  Processes check the arp table to find a translation
IP->ether. If not found, they ipc into the arp daemon to add the
entry, and block until the status of the entry changes.  The arp
daemon will retransmit the arp request until it gets a reply or times
out.  The arp daemon also responds to arp requests from other machines
in the network.\footnote{see bin/arpd/arpd.c and
libexos/net/arp\_table.c} There are basically two functions user-level
programs need to know about:\footnote{see include/exos/net/arp.h}

\begin{itemize}
\item {\tt arp\_resolve\_ip(ipaddress,ethaddress for result, interface)}
\item {\tt arp\_remove\_ip(ipaddress)}
\end{itemize}

\item{\bf Routes}: The route procedures are used to find out which is the proper
interface for the ipaddress. All process have a route table that is in
shared memory.  Each process can read and write into this table.  The
first process sets the routes by default.\footnote{see
ebin/rconsoled/setup\_net.c} The ``route'' program can also be used
for this purpose.  It is important to set the routes early on, as most
libos code uses this to locate which interface to use. More
information as well as interface definitions can be found in the
source.\footnote{see include/exos/net/route.h and
libexos/net/routenif.c}

\item{\bf DNS}:

ExOS uses the same routines used in OpenBSD.  You just have to make
sure that all the networking code is ready by the time you need to
resolve your first address.

\end{itemize}





