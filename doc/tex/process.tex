\section {ExOS Process Management}

When an application starts its library operating system must go
through what is very similar to a full OS startup in a conventional
system. This starts by calling {\tt ProcessStartup} which then does
per-subsystem initialization.\footnote{see lib/libexeos/os/process.c}
Further, the very first process to start does special
one-time-per-boot initialization for the entire system.

Sometimes different ExOS subsystems want to be able to do their own
processing when a process forks or execs (for example, the fd code
wants to close close-on-exec fd's and up the ref counts on the other
fd's every time a process forks or execs). We handle this with
call-backs. You register a function with {\tt OnExec} or {\tt OnFork}
or {\tt atexit} or {\tt ExosExitHandler}. The function will then be
called back with some useful args when the specified event
occurrs.\footnote {see lib/libc/os/oncalls.c}

\subsection{Epilogue and Prologue code}

Each process is responsible for saving and restoring its registers
when it loses and gains the processor. The kernel notifies a process
that it is about to gain or lose the processor by calling the
processes' prologue and epilogue code. The locations of this code is
set in the processes u-area.

Currently, this code does things like save and restore registers,
check for timer time-outs, and check for signals.\footnote{see 
lib/libexos/os/entry.S}

\subsection {procd}

Hector.

\subsection {Process Creation}

New ExOS processes are created through the standard POSIX {\tt fork}
and {\tt exec} calls. These are slightly complicated by the fact that
a process must duplicate itself or replace itself rather than having a
nice kernel context to do this from.

\subsection{exec}

Exec just creates a new environment, loads the new process into it,
and then kills itself.\footnote{see lib/libexeos/os/exec or
lib/libexos/os/shexec--I'm not sure which these days} The shared
library is also mapped into the upper part of the user address space.

Shared libraries are implemented pretty poorly right now. Currently,
they are just a program that has every libc function linked in. A stub
.S file is then generated that points to the location of each symbol
and a program wishing to link against the shared library just links
against this stub file. {\tt exec} then makes sure that each symbol is
really where this stub file says it is by mapping in the shared
library at the right place.

Exec must also run through all the ExOS subsystems and notify them
that an exec has occurred so that they can update their reference counts
and other data.

\subsection{fork}

Fork uses copy-on-write to speed forking. It creates a new
environment, duplicates the page tables of the parent, and marks every
non-shared entry in both environments as copy-on-write and
read-only. Then, the page-fault handler in each process handles
allocating new pages and duplicating them when a write
occurs.{\footnote{see lib/libexeos/os/cow.c}

Fork must also run through all the ExOS subsystems and notify them
that a fork has occurred so that they can update their reference counts
and other data.
