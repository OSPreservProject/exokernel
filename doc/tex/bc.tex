
\label{sec:bc}

The kernel provides a simple buffer cache that applications control.
Each physical page of memory in the system is in one of several states
depending on whether it is mapped into any env's or not and whether it
contains a disk block or not. The possible states are all possible
combinations of those two variables: empty/allocated, empty/free,
full/allocated, and full/free.

All the kernel buffer cache does is to remember what pages contain
disk blocks (whether they are free or not) and expose this mapping to
applications.  Apps can then probe the buffer cache to find which ppn
holds a given disk block and use the normal page mapping calls to then
map this page. When a page that contains a block is freed it is still
kept in the buffer cache, if the page is not dirty (if an app free's a
dirty page that means the app doesn't want its changes hanging around
so we dump the page).

Blocks can be inserted into the buffer cache in one of two
ways. Either an application can insert arbitrary pages or the app can
request the disk to read a set of blocks into the cache.

Protection is done via proxy exonodes. Before an application can
insert an arbitrary page into the cache the app must present an
exonode that says that the app has write access to the file. Only read
access is required to have the disk read a set of blocks into the
cache.

